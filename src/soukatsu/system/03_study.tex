ubsection*{勉強会総括}

%\writtenBy{\systemChief}{宇佐}{基史}
\writtenBy{\systemStaff}{宇佐}{基史}

今年度前期は,以下の一つの説明会と二つの勉強会を開催する予定だった.
\begin{itemize}
	\item 新入生サービス説明会
	\item Git勉強会
	\item サーバ構築勉強会
\end{itemize}

\begin{itemize}
\\\\
	\item{新入生サービス説明会}
\\新入生サービス説明会は,定例会議内にて新入生に本会のシステム及びサービスの周知を目的としてシステム管理局連絡の形で行われた.例年開催の事もあり,問題なく取り図らうことが出来た.しかし,コロナ禍の為もあってか,新入会員のサービス利用者数は芳しくなかったと考えられる.
	\ item{Git勉強会}
\\本年度8月1日にGit勉強会を行った.この時期選定は春セメスターの主な講義が終了し,かつ夏季休暇最初期に設定することで,参加者が都合を付けやすく開催側の準備期間も十分に設けられると考えたことに由来する.結果として多くの参加者を募ることができたため適切な判断だったと考えられる.下記にあるような改善が見込める点も多いが,「Git概要の理解」,「総会文書作成に必要な技術の把握」の二点においても概ね達成できた.又,会員におけるGit初心者の大半が参加できたこと,勉強会への参加が不可能だった未経験者のフォローが勉強会担当者個人の規模でできたこと,事後アンケート結果等から,「資料の質」,「必要者全会員への学習機会の提供」の観点でも十分な担保を確保できた勉強会だったといえる.ただ,全面オンラインでの勉強会である事に加えホスティングサービス移行に伴う作業の存在もあって多くの改善が見込める物でもあった.以下にその主とするところを記す.
\subsubsection*{開催時間の長さ}
勉強会全体で3時間超となる時間を要した.オンライン開催の為致し方ない面もあったが,改善する余地があることは否めない.他の問題の解決に付随しての短縮が見込める.参加者の負担低減の為にも時間の短縮が求められる
\subsubsection*{Texのフォロー}
総会文書の作成に不可欠なTexの解説を勉強会に内包させることができなかった.今回に限って結果として時間長化を防げたが,次回以降は例年どうりの同時並行で行いたい.
\subsubsection*{上回生への依存}
ホスティングサービス移行作業をはじめ,一部作業を\fourthGrade{}にゆだねる場面があった.当日の勉強会進行の点でも古参会員の機転とノウハウに依存する場面があった.主要回生である2回生のみでの運営が望ましい.
\subsubsection*{担当者の経験値不足}
勉強会中に起きた不足自体に担当者が対応できない場面があった.
局内事前準備の過分の負担が少なくなかった.
\subsubsection*{オンライン上の勉強会のノウハウ}
今回はZOOMのブレイクアウトルームを利用した形で勉強会を行った.混乱を避ける上で有効だったが,個人の進捗の差から生まれる待ち時間を吸収できる形態の確立が望まれる.又,あらかじめ作業環境などからブレイクアウトルームの割り振りを決定するなどの事前準備で改善が見込める問題も存在した.ブレイクアウトルームの開始と終了の判断もより良い形の模索が求められる.
\subsubsection*{メンターの意義}
具体的な手段の考案が必要だが,勉強会参加者とメンターのやり取りのいっそうの充実が望まれる.作業環境による差異や個人単位の対応,全体講義で扱えなかった付加分の知識の提供などメンターによる勉強会の充実を図れる場面が見受けられた.メンターは安定して確保できる要素ではないので状況によった判断が必要だが,今回の勉強会においてはメンターの負担配分の見直し余地があると考えられる.

	\item{サーバ構築勉強会}
\\コロナ禍に複数の勉強会の開催は困難と考え,後期に持ち越すこととした.

\end{itemize}
