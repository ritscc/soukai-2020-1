\subsection*{全体総括}

\writtenBy{\systemChief}{齋藤}{竜也}
%\writtenBy{\systemStaff}{齋藤}{竜也}

2019年度秋学期総会にて,会内方針と局内方針の二つを立てた.
会内方針は以下の三つであった.
\begin{itemize}
    \item 情報設備・システムの管理と維持
    \item 会内サービスについての知識向上
    \item 会内での快適な環境
\end{itemize}

情報設備・システムの管理と維持については,例年と違いコロナ禍での活動であったが,
本会アカウントの生成やメーリングリストの追加などリモートで対応したため,必要な業務は行えたと言える.
また,本会で使用するGitのホスティングサービスをGitLabからGitHubへの移行を行った.
コロナ禍の影響により,プロジェクト活動などもリモートで行うことが予想されたため,
有志の運用であったDiscordをシステム管理局の運用に変更し,リモートでの活動を円滑に行えるよう促した.


会内サービスにについての知識向上については,資料を作成し定例会議で周知した.Wi-FiやクライアントPCなどの
サークルルームでの利用を前提としたサービス以外は問題ないと言える.

会内での快適な環境にについては,コロナ禍の影響でサークルルームの利用が不可能であったため,
方針で立てた新たな会内サービスの提供などは取り組めなかった.

局内方針については以下の三つであった.
\begin{itemize}
    \item 局内勉強会の開催など知識向上に務める
    \item サーバ管理の徹底
    \item クライアントPCのメンテナンス
\end{itemize}
  
局内での勉強会はGitに関する勉強会を行った.詳細は勉強会総括で述べる.サーバ管理に関してはリモートで行える業務を行った.
クライアントPCのメンテナンスに関してはサークルルームの利用がなかったため,行っていない.
