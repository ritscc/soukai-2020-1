\subsection*{プロジェクト活動総括}

%\writtenBy{\president}{八木田}{裕伍}
%\writtenBy{\subPresident}{八木田}{裕伍}
%\writtenBy{\firstGrade}{八木田}{裕伍}
\writtenBy{\secondGrade}{八木田}{裕伍}
%\writtenBy{\thirdGrade}{八木田}{裕伍}
%\writtenBy{\fourthGrade}{八木田}{裕伍}

\subsubsection*{全体総括}
2020年度春学期のプロジェクト活動は,5月中旬頃から企画書の募集を開始し,6月中旬に活動を開始した.
各プロジェクトには活動ごとに週報を提出することを義務付け,進捗確認を行った.
全6個のプロジェクト全てが設立された.
2020年度春学期は新型コロナウイルス感染症流行のため,全ての活動をオンラインでの開催とした.
それに加えて,オンサイトでの追い込み合宿やプロジェクト発表会の開催が難しいことから,
これらを行わないこととした.
これによって,プロジェクトの目標が達成されないまま終了する事態を避けるため,
2020年度は全てのプロジェクトを通年プロジェクトとし,春学期終了時には報告書の提出も行わないこととした.

以下に2020年度春学期に活動していたプロジェクトの一覧を示す.

\begin{itemize}
\item AlphaZero班
\item DTM班
\item Unity班
\item 競馬AI班
\item 自然言語処理班
\item C言語班
\end{itemize}

プロジェクト活動の総括は以下の六つに分けて行う.

\begin{itemize}
\item 目標の総括
\item プロジェクトの内容
\item 週報
\item 報告書
\item 追い込み合宿
\item プロジェクト発表会
\end{itemize}

\subsubsection*{目標の総括}
2020年度春学期の目標は以下の三つであった.

\begin{itemize}
\item 活動を通して技術力の向上を図る
\item 個人のみならずグループ活動としての経験を得る
\item 活動によって得られた成果を本会Webサイトを通して公開する
\end{itemize}

これらを踏まえた総括を以下に記す.

活動を通して技術力の向上を図るに関しては,
プロジェクト活動をオンラインのみでの活動に限定していたことから,
プロジェクト進行の遅延や,
リーダーが班員の技術力を把握できなかった事態が見受けられた.
しかし,程度に差はあれど,会員のほぼ全ての技術力が向上したことから,
この目標は概ね達成できたと言える.

集団行動の重要性を学ぶに関しては,
毎回のプロジェクト活動に参加率は高く,達成できていたと言える.
しかし,途中からグループでの活動をせず,個人活動のみとなっていた班も存在した.

得られた成果を本会Webサイトを通して公開するに関しては,
2020年度春学期はプロジェクト活動報告書を作成しなかったため,達成できなかった.

\subsubsection*{プロジェクトの内容}
プロジェクトの内容については,全ての班において適切であった.
また,類似した内容のプロジェクトがあったため,リーダー間で話し合って統合した.

\subsubsection*{週報}
今学期から研究推進局内のみではなく,上回生会議においても週報の確認を行ったことにより,提出率が向上した.
研究推進局の週報担当がリマインドを忘れる事態があったため,局内での対応が必要である.

また,2019年度秋学期方針にて,週報が出ていないまたは週報に活動の継続が難しい旨が記述されていた場合,
プロジェクトのリーダーを上回生会議に招集し,
プロジェクトの存続を問うこととしており,
2020年度春学期ではUnity班が該当したが,
事態の発生が春学期の活動の終盤であったため,上回生会議に招集できなかった.

\subsubsection*{報告書}

報告書の提出は,
2020年度は通年プロジェクトのみであり,活動がオンラインであったため,春学期は行わないこととした.

\subsubsection*{追い込み合宿}
追い込み合宿は,
2020年度は通年プロジェクトのみであり,活動がオンラインであったため,春学期は開催しないこととした.

\subsubsection*{プロジェクト発表会}
プロジェクト発表会は,
2020年度は通年プロジェクトのみであり,活動がオンラインであったため,春学期は開催しないこととした.

