

\subsection*{新歓総括}

% 執行委員長としてWeb交流会へ参加したため,執行委員長としての文責になります
\writtenBy{\president}{原}{佑馬}

2020年度春学期の新歓の目的は,以下の2点であった.

\begin{itemize}
    \item 新入生に会の活動内容や活動方針について知ってもらう
    \item 新入生に会に興味を持ってもらう
\end{itemize}

これらの目的を達成するため,以下の4点の目標を掲げた.

\begin{itemize}
    \item 企画に対して参加してもらう
    \item 気軽にサークルルームに来てもらう
    \item 新入生に本会でやりたい事を見つけてもらう
    \item 新入生の中長期的な定着
\end{itemize}

これらの目標を達成するため,一次企画と二次企画の企画を行った.

結果的に,コロナウイルス感染拡大防止のため新歓が中止となり,これらの企画の実行も中止したが,
新歓中止の対策として大学が開催したWeb交流会に参加することで,新歓の目的の達成を図った.

以下に,一次企画及び二次企画の概要と,Web交流会の概要と結果を示すことで,新歓総括とする.

\subsubsection*{一次企画概要}
集まった新入生に対し,会の紹介やWelcomeゼミの紹介,2020年度春学期プロジェクトリーダーによるLTを行う事によって,目標の達成を図る.

\subsubsection*{二次企画概要}
上回生のサポートの下,新入生がProcessingを用いたゲーム制作を行うワークショップを実施することによって,目標の達成を図る.

\subsubsection*{Web交流会}
Web交流会は,新入生と上回生の歓迎・交流機会を担保するものとして大学が主催したオンライン交流会である.
Zoomを用いて4月と6月の2回行われ,参加団体は60分の団体紹介時間が与えられた.

本会は,4月と6月の両方に参加し,担当者である執行委員長が当日の発表を行った.

本来,Web交流会のような新歓に関わる行事は新歓担当が行うべきである.
しかし,Web交流会は団体の代表者が参加するものであったため,担当者は会を代表する執行委員長が適任であると判断した.

Web交流会においては,普段の活動内容の紹介やプロジェクト活動の紹介,Welcomeゼミの紹介を行った.
結果多くの新入生がWelcomeゼミへ参加し,また本会の活動に興味を持って入会したため,Web交流会の参加により新歓の目的を達成することができたと言える.
