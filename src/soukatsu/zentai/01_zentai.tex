\subsection*{2020年度春学期活動総括}

\writtenBy{\president}{原}{佑馬}

本会の目的である「情報科学の研究,及びその成果の発表を活動の基本に会員相互の親睦を図り,学術文化の創造と発展に寄与する」ことを達成するため,方針として以下の五つを立てた.
これらについてそれぞれ評価を行うことで2020年度春学期の総括とする.

\begin{itemize}
    \item 親睦を深める
    \item 規律ある行動
    \item 自己発信力の向上
    \item 会員間の技術向上
    \item 外部への情報発信
\end{itemize}

\subsubsection*{親睦を深める}
    2020年度春学期活動では,Welcomeゼミや新歓交流会,プロジェクト活動を実施することによって会員間の親睦を図った.

    尚,2020年度春学期活動においては何れの活動も,コロナウイルス感染拡大防止のためオンライン形式で行った.

    Welcomeゼミは,新入生と上回生が親睦を深める重要な機会であった.
    オンライン形式での実施となったため,直接ペアを組んでいない新入生と上回生が親睦を深めることは困難であったが,
    対面での活動ができない中で充分親睦を深めることができた.

    新歓交流会にも多くの新入生が参加し,自己紹介後には交流の機会も設けた.
    上回生との会話を通して,交流会前後では良い影響があったように思われる.

    プロジェクト活動では,各々の班で共同開発や発表が活発に行われていたが,
    オンライン形式である都合上,会員同士の交流の機会はやや少なかった.

\subsubsection*{規律ある行動}
    本項では,遅刻・欠席連絡と備品整備,サークルルームの使用方法の三つについて評価する.

    遅刻・欠席連絡については,概ねSlackの専用チャンネルにおいて行われていたが,下回生の模範となるべき上回生による無断欠席や,開始時刻を過ぎてからの連絡が少なからず見受けられた.

    備品整理及びサークルルームの使用方法については,コロナウイルス感染拡大防止のためサークルルームが使用不可であったため,備品の使用や入室ができず,問題は発生しなかった.

\subsubsection*{自己発信力の向上}
    自己発信力の向上の機会として,2020年度春学期活動では,LTを行った.

    LTでは,割り当てられていた会員の全員が発表した他,有志による発表も多く,定例会議での発表は充実していた.

    尚,例年自己発信力の向上の機会として実施しているプロジェクト発表会に関しては,プロジェクト活動の活動形態の都合上,半年間での目標の達成が困難であり,
    活動報告によって会員への負担が大きくなることが予想されたため,2020年度春学期は実施しなかった.

\subsubsection*{会員間の技術向上}
    会全体の技術力を向上させることを目的として,LTやプロジェクト活動,勉強会を開催した.

    LTでは,新入生を含めた有志での発表が多く,また質も非常に高かったため,会員の技術力に良い影響を与えていたと考えられる.

    プロジェクト活動や勉強会も,オンライン形式で実施されたが問題はなく行われた.

\subsubsection*{外部への情報発信}
    会外へ活動を発信する機会として,主に本会Webサイトと会公式Twitterが挙げられる.

    本会Webサイトでは,新歓に関する告知やスライド資料を掲載し,会公式Twitterでは,LTやイベントが行われる度にその様子が発信された.
    これらの頻度や内容は適切であり,本会の活動を知ってもらうためには充分であったと思われる.
