\subsection*{運営総括}

\writtenBy{\subPresident}{西見}{元希}

2020年度春学期の運営に関して,以下の4点から総括を述べる.
\begin{itemize}
  \item 定例会議
  \item 上回生会議
  \item 局
  \item 企画
\end{itemize}

\subsection*{定例会議}
対面での実施が不可能であったため,毎週木曜日にZoomで実施した.
内容は例年と変わらず局連絡および会員によるLTを行った.
参加人数も大きく減ることはなく,概ね例年通りであった.
これによって発生したトラブルなどもなく,
コロナ禍の状況下でも普段通りの活動を行うことができていたと言える.

Slackの専用チャンネルによる議題管理は行われなかったが,
上回生会議の議事録において十分な議題管理は遂行されていたため,問題はなかったと思われる.

会内向けのフォームは2019年度秋学期と同様にすべて「RCCフォームビューア」を使用したが,
管理の引き継ぎも円滑に行われ,締め切りを超過する回答も見られなかったため,
上手く活用できたと考える.

総じて,対面での活動が制限されている中でよく活動できていたと言えるであろう.

\subsection*{上回生会議}
毎週水曜日にZoomで問題なく開催された.
なお,開催曜日と時間帯に関しては,主な出席者である執行部の都合を擦り合わせ設定した.
一部の執行部が無断欠席する場面も多少見られたが,
リモート授業で曜日感覚がなくなっていることもあり,ある程度仕方ないことであったと思われる.
また,そういったときに代理人がすぐに出席していたり,
逆に欠席者が前回の会議の内容を把握していなかったという場面もなかったため
大きな問題は生じなかった.

議題の企画書の提出に関しては,方針としては前日までの提出を義務としていたが
昨今の特殊な状況下でそれを遂行するのは難しく,担当者への連絡としては
次の上回生会議までに企画書を提出するように,という形式で催促することが殆どであった.
それに対して,提出締め切りに遅れた企画書は一つもなかったため非常に評価できるであろう.

上回生会議で取り扱った大きな議題としては,
課外自主活動の対面活動の制限に対しどのような形式で本会の活動を行っていくかというものがあった.
特に,入会手続き,新歓企画,定例会議およびプロジェクト活動は例年と大きく状況が異なったために,
上回生会議での話し合いによってオンラインにおける本会の活動を支えることができたのではないかと考える.
%もう二度とやりたくない…

\subsection*{局}
各局は議題がほとんど存在しなかった渉外局を除き概ね局会議を遂行することができていた.

春学期に予定していた局配属であるが,
予定していたタイミングではまだ新入生が本会に対して十分に馴染めておらず,
また,リモート活動しか経験できない\firstGradeに対し
適切に各局の業務内容を説明した上で局配属希望調査や面談を実施するのは早急であるという結論に至り,
2020年度は秋学期に実施することとした.


\subsection*{企画}
企画の進捗確認に関しては,上回生会議にて担当者からの報告を随時受けており,執行部内である程度共有された.
春学期に行った企画には新歓・Welcomeゼミ・新入生歓迎会があったが,どれも\secondGradeが主体となって企画し,
担当者が殆どの作業を行っていた.
そのため,例年と企画内容が大きく異なるにもかかわらず問題なく企画を遂行できた.

その反面で,担当者に業務が集中してしまったという側面もあり,
情報共有が十分に行われていたとは言い難い企画もあった.

また,KPTも行われていなかった企画も存在し,
もう少しこの特殊な状況下での企画内容を後の代に伝える努力をすべきであろう.


