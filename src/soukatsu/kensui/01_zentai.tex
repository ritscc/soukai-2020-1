\subsection*{全体総括}

\writtenBy{\kensuiChief}{八木田}{裕伍}
%\writtenBy{\kensuiStaff}{八木田}{裕伍}


2020年度春学期の研究推進局は以下の3点を目的として活動を行った.
\begin{itemize}
\item 平常活動の支援
\item 会員が興味関心のある活動ができる環境づくり
\item 発信力を養うための環境づくり
\end{itemize}

\subsubsection*{平常活動の支援}
平常活動の支援に関しては,プロジェクト活動の進捗管理やサポートを行った.
プロジェクト活動の進捗管理では,活動で生じた問題を週報で抽出し,上回生会議の議題に上げた.
活動の後半において,Unity班がプロジェクト活動を実施せず,
期限までに週報の提出がされない事態が発生したが,春学期も終盤であったことから上回生会議への招集ができなかった.
その他の班については,週報で報告された問題も無く,プロジェクト活動での支障も無かった.
2020年度春学期はプロジェクト活動はオンラインのみであり,
かつ追い込み合宿やプロジェクト発表会を開催しなかったため,部屋取りなどの業務は行わなかった.

\subsubsection*{会員が興味関心のある活動ができる環境づくり}
会員が興味関心のある活動ができる環境づくりは,十分行えなかったと考えられる.
原因として,毎年開催していた勉強会を,2020年度春学期には対面での実施が難しいことから開催しなかったことが挙げられる.

\subsubsection*{発信力を養うための環境づくり}
発信力を養うための環境づくりに関しては,毎週の定例会議でLTを行った.
毎週の定例会議の時点で次週以降の担当者を通知したため,LTのための準備の時間を与えられたと考える.
2020年度春学期には,ほぼ全てのLT担当者が担当週までにLTを行うことができた.
理由としては,局員がメールやその他Twitterなどのツールを活用して,積極的にリマインドを行ったことが挙げられる.

また,2020年度春学期は,LTアンケートの実施から結果の公表まで円滑に実施することが出来た.
これは,研究推進局が定例会議などのスケジュールを把握して事前に行動していたためである.
