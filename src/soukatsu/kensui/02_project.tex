\subsection*{プロジェクト活動総括}

%\writtenBy{\kensuiChief}{中尾}{龍矢}
%\writtenBy{\kensuiStaff}{中尾}{龍矢}

本局の方針は,会員のプロジェクト活動を円滑に進行する為,進捗確認や活動場所確保などのサポートを行うことである.

\subsubsection*{企画書の募集}

プロジェクト活動のリーダーは基本的に\secondGrade{}が勤め,企画書は計七つ集まった.
そのうち「Unityで遊ぼ班」と「AR班」は学習内容が似通っていたため,話し合いにより二つの班を「Unity班」に統合した.
結果としてプロジェクト数は六つとして活動を開始した.

\subsubsection*{週報の回収・催促}

プロジェクトの進捗を管理する目的で各プロジェクトリーダーは週報の提出を義務付けられている.
週報の回収にはGoogleフォームが用いられ,提出が停滞しているプロジェクトにはSlackを通して提出を催促した.
今学期から週報の催促を上回生会議でも行うようにしたためか,完全オンラインの活動にも関わらず週報の提出率は例年よりも良いものであった.
しかし,週報担当の催促忘れや,不適切な理由による活動未開催があったが活動後半に発生したため対応ができなかったなどの問題も見られた.

\subsubsection*{会員のプロジェクト管理}

本局では,各会員がどのプロジェクトに所属しているかを把握し,プロジェクトが途中で終了した場合などに所属していた会員のプロジェクト異動などを管理している.
コロナによる課外活動の制限などにより,全プロジェクトを通年にしたり,全面的にリモートでの活動に切り替えたために様々な問題が発生していた.
グループでの活動よりも個人での学習時間が多く,班員の技術力の向上が懐疑的である問題点も見られた.
それに反し,複数の班が同じzoomの部屋に集まり相談しながら活動をするなど,班を跨いで協力しあう面も見られた.
途中から参加した新入生にはプロジェクトリーダーとの連絡により無事編入を行った.
異動はなかった.

\subsubsection*{発表の機会の提供}

例年はプロジェクト活動の成果発表をプロジェクト発表会にて行ってきたが,2020年度春学期は全プロジェクトが通年かつオンライン実施のため,プロジェクト発表会,追い込み合宿,報告書の作成を行わなかった.
プロジェクト発表会は2020年度秋学期に行う見通しである.
