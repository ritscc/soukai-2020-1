\subsection*{\firstGrade{}総括}

%\writtenBy{\firstGrade}{星名}{藍乃介}
\writtenBy{\secondGrade}{星名}{藍乃介}
%\writtenBy{\thirdGrade}{星名}{藍乃介}
%\writtenBy{\fourthGrade}{星名}{藍乃介}


\firstGrade{}総括にあたり,
本会の活動について未だ十分に把握していない\firstGrade{}が総括を行うことは難しいと判断したため,
例年通り上回生が総括を行った.


\subsubsection*{会員間の親睦}
今年度の活動はオンラインであっため,
コミュニケーションは例年に比べ,特に取りづらい状況であった.
そんな中でも,新歓交流会では,普段交流の少ない会員間での親睦を深めることができたと考えられる.
また,平常活動でも,TwitterやDiscodeなどSNSを通して,
\firstGrade{}同士だけでなく,上回生と交流を行っていた\firstGrade{}も多数見受けられた.


\subsubsection*{自己発信力の向上}
何人かの\firstGrade{}が,2020年度春学期では任意であるLTを行った.
さらに,普段からの勤勉さを伺えるような,高いクオリティの発表も多数見て取れた.
\firstGrade{}の,自ら進んで知見を得てそれを発表,会内で共有しようとするなどの高い積極性が伺える.


\subsubsection*{技術力向上}
プロジェクトやWelcomeゼミ,git勉強会などでの活動を通して向上したと考えられる.
