\subsection*{プロジェクト活動方針}

%\writtenBy{\kensuiChief}{中尾}{龍矢}
\writtenBy{\kensuiStaff}{中尾}{龍矢}

本項では本局におけるプロジェクト活動業務に関する2020年度秋学期の方針を以下の点において述べる.

\begin{itemize}
\item 企画書の募集
\item 週報の回収・催促
\item 会員のプロジェクト管理
\item 発表の機会の提供
\item 報告書の管理
\end{itemize}

\subsubsection*{企画書の募集}

2020年度春学期に設立されたプロジェクトはすべて通年のため,2020年度秋学期も継続して活動していく.
新規プロジェクト設立については,新型コロナウイルス感染症に対する大学の対応やリモート環境などの安定していない状況によって,活動の継続が困難になることを懸念して設立は認めない.

\subsubsection*{週報の回収・催促}

2020年度春学期と同様,活動後の提出を義務とし,Slackでの催促に加えて上回生会議でも週報の確認を実施して提出を確実にする.
週報の提出状況を管理する際,週報が何週目の提出なのか,一週間の間の何回目の活動なのかが分かりづらいという問題があった.
そのため,より細かく管理できるようにフォームに「活動週」を「何週目の活動ですか?」に換え,「週のうち,何回目の活動ですか?」を追加する.
プロジェクトの解散については,班員が3人未満もしくはリーダーが欠けた場合,そのプロジェクトは趣意書をもって理由を記述したのちに,上回生会議によって解散される.

\subsubsection*{会員のプロジェクト管理}

2020年度春学期後半で発生した,不適切な理由による活動未開催な班については,リーダーを上回生会議に呼び出し,存続の意思を問う.
無い場合,班員を呼び出した後に研究推進局の下,上回生会議にて適切な処理を行う.
ある場合は,班内での活動計画を再考後,上回生会議にて提出を行う.

\subsubsection* {発表の機会の提供}

2020年度秋学期にはプロジェクト発表会を行う.
その際,事前に配布された報告書をその場で読む時間を設ける,スライド発表,質疑応答との形式で執り行う.
ただし,対面で行うかオンラインで行うかは状況を観察しながら決定する.
報告書のテンプレートを各プロジェクトリーダーに配布,PDFでレビューした後,対面で行うのであれば2部印刷する.なお,テンプレートの配布の時期は定例会議にて連絡を行う.
追い込み合宿は,10月中に外出を伴う課外自主活動自粛要請が無効になった場合に実施する.

\subsubsection* {報告書の管理}

作成された報告書はプロジェクト表会中にレビュー,修正し,修正期間を経た後にPDFで提出してもらう.集めたPDFは渉外局に依頼し,本会Webサイトに公開する.
報告書では載せられない制作物は,上回生会議にて著作権などの確認を行い,Webサイトでの公開が推奨される.
