\subsection*{全体方針}
\writtenBy{\soumuChief}{奥川}{莞多}
後期の活動における方針はサークルルームの使用状況によって変わる.サークルルームが使用できる場合は,「サークルルームの美化」と「備品の管理の徹底」を2020年度春学期に引き続き方針とする.一方,部室が使用できない場合は,前述の方針を取り消し,会内行事を円滑に進めることを方針とする.

\subsectin*{風紀業務方針}
\writtenBy{\soumuChief}{奥川}{莞多}
昨年度に引き続き,口頭での注意や呼びかけを主に活動する.部室が使用できない場合は,前述の方針を取り消す.

\subsection*{掃除業務方針}
\writtenBy{\soumuChief}{奥川}{莞多}
ごみ箱の各ルールについては昨年度に引き続き行う.大掃除については昨年度同様セメスターごとに1度決行する.事前告知を行うことで参加人数を確保する.部室が使用できない場合は,前述の方針を取り消す.

\subsection*{書記業務方針}
\writtenBy{\soumuChief}{奥川}{莞多}
前期において問題なく進めることができた定例会議の書記業務をしっかりと行う.総務局内に対応できる人員がいない場合は予め初期の代理を用意する.議事録についてお知らせも局内で当番を割り振り,業務を怠らないようにする.

\subsection*{会内行事方針}
\writtenBy{\soumuChief}{奥川}{莞多}
企画の告知と参加の勧めを早めにすることを心がける.また「もし参加募集期間を過ぎてからの申し込みがあった場合,定員不足,また定員から欠員が出た場合のみ参加を許可する」という原則を徹底することとする.

\subusection*{備品業務方針}
\writtenBy{\soumuChief}{奥川}{莞多}
「延滞者への通知」,「私物の明確化」を徹底する.延滞者への通知に関しては引き続きSlackでの通知を行い,早期の備品返却を呼びかける.「私物の明確化」においても備品が整理された状態を保つために管理を怠らないようにする.部室が使用できない場合は,前述の方針を取り消す.

%\writtenBy{\soumuStaff}{奥川}{莞多}
