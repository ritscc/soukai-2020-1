\subsection*{全体方針}

%\writtenBy{\kaikeiChief}{田邊}{雄士}
%\writtenBy{\kaikeiStaff}{田邊}{雄士}

\writtenBy{\kaikeiStaff}{田邉}{雄士}

本局は,本会の財務と財政を管理し,責任を持って活動することを2020年度秋学期方針とする.


\subsection*{財務方針}

\subsubsection*{局会議}
春学期に引き続き週に一度局員が集まる日を設け,会議を行う.

\subsubsection*{会計情報の公開}
定例会議及び上回生会議の際にGoogleドライブ上に情報を公開する.
また,より詳細な情報に関しては,必要に応じて局員が情報を公開する.

\subsubsection*{引き継ぎ文書}
作成した引き継ぎ文書に適宜変更と修正を加える.

\subsubsection*{購入申請}
基本的には受け付けない.
対面での活動が再開された場合は購入申請を受け付ける.

\subsubsection*{会費}
特別議案が通った後に,4500円を口座振込で徴収する.


\subsection*{財政方針}

\subsubsection*{予算}
局会議や上回生会議において予算執行について審議し適切に予算を執行する.

\subsubsection*{予算配分}
各企画及び,購入申請に対して適切に予算の配分を行う.


\subsection*{ハッカソン方針}
春学期は新型コロナウイルス感染症の影響を鑑みて開催しなかった.秋学期についても状況がよくならない限りは開催しない.

下記は開催できた場合の方針である.

\subsubsection*{目的}
\begin{itemize}
    \item 会員の技術力の向上と会員間の親睦を深める.
    \item 合宿形式での短期開発を経験する.
\end{itemize}

\subsubsection*{手法}
事前にフォームで会員よりテーマのアイデアを募集した後,参加者の募集と共に最終決定を行う.また,参加者の不足と参加申請の漏れを防ぐ目的で会員へのリマインドを積極的
に行う.
グループ分けは,担当者が行い,開催1週間前までにテーマと共に参加者に通知する.そこから
開催まではアイデアソンを自由に行ってよいものとする.こうすることにより,ハッカソン当日は
即時に開発を始めることも可能である.また,各グループ内で事前に開発に用いる言語の勉強など
を行うことも可能,したがって,班員同士の技術的格差の解消が見込める.加えて,グループごとにリー
ダーを設定し,グループ内でのアイデアソンや方針の決定を主導してもらう.これにより,「会員間
の親睦を深める」の目的を達成するねらいもある.その後各グループが開発を行い,最終日にその
成果発表を行う.場所はエポック立命21で行い,期間は3日間とする.
発表に関しては,所要時間は1~2時間とし,一班あたりの上限時間は20分程度とする.成果物
発表のためのスライド作成などは各班ごとに任意である.
本会のOBなど,学内の方が訪問してきた場合は特に拒まず,活動に参加させる.しかし,他大
学の方がハッカソンに参加したいと申し出た場合,担当者の事務処理が煩雑なものになりうるた
め,開催日までの時間などを鑑みて上回生会議で議論し,参加の許可を決定する.




