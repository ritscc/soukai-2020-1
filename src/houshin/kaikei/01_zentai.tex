\subsection*{全体方針}

%\writtenBy{\kaikeiChief}{田邊}{雄士}
%\writtenBy{\kaikeiStaff}{田邊}{雄士}

\writtenBy{\kaikeiStaff}{田邉}{雄士}

本局は,本会の財務と財政を管理し,責任を持って活動することを2020年度秋学期方針とする.


\subsection*{財務方針}

\subsubsection*{局会議}
春学期に引き続き週に一度局員が集まる日を設け,会議を行う.

\subsubsection*{会計情報の公開}
定例会議及び上回生会議の際にGoogleドライブ上に情報を公開する.
また,より詳細な情報に関しては,必要に応じて局員が情報を公開する.

\subsubsection*{引き継ぎ文書}
作成した引き継ぎ文書に随時変更と修正を加える.

\subsubsection*{購入申請}
基本的には受け付けない.
コロナが収束し,活動が再開された場合は購入申請を受け付ける.

\subsubsection*{会費}
特別議案が通った後に,4500円を口座振込で徴収する.(総会文書に書く前に確認)


\subsection*{財政方針}

\subsubsection*{予算}
局会議や上回生会議において予算執行について審議し適切に予算を執行する.

\subsubsection*{予算配分}
各企画及び,購入申請に対して適切に予算の配分を行う.


\subsection*{ハッカソン方針}
春学期はコロナのため開催しなかった.秋学期についても状況がよくならない限りは開催しない.

下記は開催できた場合の方針である.

\subsubsection*{目的}
ハッカソンとは,会員の技術力の向上と会員間の親睦を深めるために,合宿形式での短期開発を経験して,最終日においてその成果を発表し知識を共有する活動である.

\subsubsection*{目標}
\begin{itemize}
    \item 技術向上
    \item 集団開発の仕方を学ぶ (Gitなど)
    \item 会員間の親睦を深める
    \item 開催
    \begin{itemize}
        \item 一定の参加会員の獲得
    \end{itemize}
    \item 各班で制作物を必ず一つ制作し発表
\end{itemize}


\subsubsection*{回数}
春期休暇中に1回行う.

\subsubsection*{期間}
2泊3日

\subsubsection*{場所}
エポック立命21

\subsubsection*{開催までの流れ}
[数週間前]	テーマと参加者と開始時間を募集
           開発環境の構築や言語について学習する (任意)
[1週間前]	テーマと班,開始時間を発表
[当日以前]	アイデアソンを前もって行う (任意)
[ 当日 ]	各班はテーマに沿って開発作業 (成果物作成) を3日間で行う

\subsubsection*{開催中のスケジュール}
初日の発表までにアイデアソンを行い,各班のアイデアを発表する.
最終日には成果物の発表を行う時間を設け,それまでに各班が開発を行う.
発表に関しては,所要時間は1~2時間とし,1班あたりの上限時間は20分程度とする.進行役が必要であれば,担当者が行う.
成果物発表のためのスライド作成などは必須ではなく,各班ごとに任意である.


\subsubsection*{上回生・OB・企業の方}
参加者以外の方が活動の見学に来た場合は特に拒まない.
しかし,本会のOB・OGや他大学の方がハッカソンに参加したいと申し出た場合,担当者の事務処理が煩雑なものになりうるため,開催日までの時間などを鑑みて上回生会議で議論し,参加の許可を決定する.


ハッカソンは技術力の向上及び会員間の交流に役立つものであるため,2021年度も夏期と冬期に1回ずつ行う.
事前にフォームで会員よりテーマのアイデアを募集した後,参加者の募集と共に最終決定を行う.
また,参加者の不足と参加申請の漏れを防ぐ目的で会員へのリマインドを積極的に行う.


グループ分けは,担当者が行い,開催1週間前までにテーマと共に参加者に通知する.
そこから開催まではアイデアソンを自由に行ってよいものとする.
これにより,ハッカソン当日は即時に開発を始めることも可能である.
また,各グループ内で事前に開発に用いる言語の勉強などを行うことも可能,したがって,班員同士の技術的格差の解消が見込める.
加えて,グループごとにリーダーを設定し,グループ内でのアイデアソンや方針の決定を主導してもらう.これにより,「会員間の親睦を深める」の目的を達成する狙いもある.
その後各グループが開発を行い,最終日にその成果発表を行う.
場所はエポック立命21で行い,期間は3日間とする.

発表に関しては,所要時間は1~2時間とし,1班あたりの上限時間は20分程度とする.進行役が必要であれば,担当者が行う.
成果物発表のためのスライド作成などは必須ではなく,各班ごとに任意である.

本会のOBやOGなど,学内の方が活動の見学に来た場合は特に拒まない.
しかし,OB・OGや他大学の方がハッカソンに参加したいと申し出た場合,担当者の事務処理が煩雑なものになりうるため,開催日までの時間などを鑑みて上回生会議で議論し,参加の許可を決定する.

