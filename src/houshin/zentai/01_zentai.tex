\subsection*{2020年度秋学期活動方針}

\writtenBy{\president}{原}{佑馬}

本会の目的である「情報科学の研究,及びその成果の発表を活動の基本に会員相互の親睦を図り,学術文化の創造と発展に寄与する」を踏まえ,以下の五つを方針とする.

\begin{itemize}
    \item 親睦を深める
    \item 規律ある行動
    \item 自己発信力の向上
    \item 会員間の技術向上
    \item 外部への情報の発信
\end{itemize}

\subsubsection*{親睦を深める}
    会員間での親睦を深めることは,本会の目的の一つである.
    サークル活動は個人で成り立つものではなく,会員が互いに手を取り合い,助け合うことで実現される.
    そのため,会員間の親睦を深めることはサークル活動を成り立たさせる上で必要不可欠である.

    本会の主な活動であるプロジェクト活動では,各プロジェクト内での密な連携が要求される.
    2020年度秋学期においてもコロナウイルス感染拡大防止のためオンライン形式で活動を行う必要があることが予想されるが,
    そのような場合でも積極的にコミュニケーションを取り合うよう心掛ける.

    加えて,クリスマス会などといったイベントを企画し,会員間の親睦を深める機会を設ける.

\subsubsection*{規律ある行動}
    サークルを運営していく上で一定の規律は必要である.
    本項では,サークルが適切かつ円滑に運営されるために,会員が最低限行うべき行動方針を二つ示す.

    一つ目に,遅刻及び欠席連絡について述べる.
    遅刻・欠席は可能な限りしないことが望ましい.
    しかし,やむを得ない事情がある場合は,その理由と遅刻であるならば到着予想時刻を明記した上で連絡すべきである.
    その際,不明瞭な理由を記載することは避け,会全体に伝えるべき内容でない場合は,執行委員長やイベントの主催者に直接伝えることを心がける.
    また,これらの連絡は,遅刻・欠席することが判明した場合に早急に行うことを心がける.

    二つ目に,会内ルールの徹底について述べる.
    2020年度秋学期においてもコロナウイルス感染拡大防止のため,サークルルームの使用が制限されることが予想される.
    大学の許可を得てサークルルームを使用する場合には,必要以上の備品の持ち出しや移動のような他の会員の迷惑になる行為は慎まれるべきである.
    また,ゴミの不始末などサークルルームの環境悪化の原因となる行為もあってはならない.
    会員は他会員に対する礼儀及びマナーとして,このような行為の無いように心がけることとする.

\subsubsection*{自己発信力の向上}
    自らの意見を相手方に対してわかりやすく伝える能力を身につけることを目標とする.
    2020年度秋学期にはLTやプロジェクト発表会などにおける発表と,会誌や活動報告書,Advent Calendarにおける文書作成を通して相手に自分の意見を伝える機会を数多く設ける予定である.
    しかし,これらの活動において相手が理解し難い内容となってしまうような事態は回避すべきである.
    そこで,行事における発表に関しては,会員間でレビューを行い,プレゼンテーション能力の向上を図る.
    また,文書の作成についても会員間で校正を行い,正しい文書作成能力を身につけることを目指す.
    上記の活動を通し,聴衆や読み手に簡潔かつ正確に伝える能力の向上を図る.

\subsubsection*{会員間の技術向上}
    会員間で技術を共有し合うことで,会員が新たな分野に触れる機会を提供すると共に,会全体の技術力を向上させる.
    2020年度秋学期には,LTやプロジェクト活動の他にも勉強会などを実施することによって,技術向上につながる多くの機会を提供していく.

\subsubsection*{外部への情報の発信}
    会外へ活動を発信し,本会の活動を知ってもらうことが目的である.
    会公式 Twitter を主軸に本会 Web サイトも活用することで,より多くの人に本会の活動を知ってもらえるよう,積極的に情報を発信していく.
    また,発信する内容は,知識がある人,そうでない人も楽しめるよう伝え方や内容を意識する.
    一方で,会誌での会員コラムなど,各会員の自由な内容を発信できる場も提供することで,「情報発信の楽しさ」と「伝え方の技術」の両方から力をつけられるよう心がける.
