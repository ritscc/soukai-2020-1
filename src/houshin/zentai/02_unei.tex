\subsection*{運営方針}

%\writtenBy{\president}{西見}{元希}
\writtenBy{\subPresident}{西見}{元希}
%\writtenBy{\firstGrade}{西見}{元希}
%\writtenBy{\secondGrade}{西見}{元希}
%\writtenBy{\thirdGrade}{西見}{元希}
%\writtenBy{\fourthGrade}{西見}{元希}

2020年度秋学期の運営について,以下の4点から方針を述べる.
\begin{itemize}
    \item 定例会議
    \item 上回生会議
    \item 局
    \item 企画
\end{itemize}

\subsubsection*{定例会議}
秋学期においても,春学期同様週1回の定例会議を行う.
状況次第ではあるが,リモート開催を基本として開催する予定である.

必要があればSlack内の専用チャンネルに自由に議題投稿を行いそれに沿った形での議事進行に努め,
局や企画からの連絡や会員全体ですべき議決,LTなどを行う.


\subsubsection*{上回生会議}
秋学期においても,週1回の頻度にてオンラインで上回生会議を行う.
執行部および提出された議題関係者は全員参加とし,欠席の場合は必ず代理人を立てるようにする.

通常通り,議決権のない会員に関しても参加の意志があればその出席を認めることとする.
特に,\firstGrade{}に対して上回生会議に参加可能である旨を周知させる.

会議で取り扱う企画書の提出は遅れないようにし,企画書のレビュー時間の短縮と円滑な議事進行に努める.
各局長は局員が上回生会議における議題の内容を把握できるように努める.


\subsubsection*{局}
春学期に行うことのできなかった局配属を秋学期に実施する.
秋学期の本会の活動が開始次第,各局の業務内容について十分な説明を行ったうえで希望調査を実施し,
その結果に応じて面談を実施する.

局長は局会議を実施し,上回生会議内での議題を共有し局員が同局に関する議題内容を把握できるようにする.
また,対面での活動の可否に関わらず今後の運営に支障が出ないよう引き継ぎを行っていく.

\subsubsection*{企画}
本会外部との関わりがある行事に関しては春学期に設定した担当者がその企画を行う.
各企画について2人以上担当者をおくようにし,もし担当者が退会などの理由によって活動できなくなった場合は
新たに担当者を追加するか他の会員でフォローを行し,1名に負担がかかるのを避ける.
企画書提出から企画終了までは進捗確認のため担当者は上回生会議にてその度合いを報告する.

また,企画終了後には振り返りとしてKPTを行い,引き継ぎ資料の作成に努める.
企画担当が局である場合はその局の局会議内で行う.
そうでない場合は参加者に応じて担当者のみ,定例会議内,
上回生会議内のうち適切な場で実施する.
企画終了後に時間がある場合はその場でKPTを行うように心がける.

