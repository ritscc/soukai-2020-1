\subsection*{\newGradeIfKouki{}\secondGrade{}方針}

%\writtenBy{\firstGrade}{宇佐}{基史}
\writtenBy{\secondGrade}{宇佐}{基史}
%\writtenBy{\thirdGrade}{宇佐}{基史}
%\writtenBy{\fourthGrade}{宇佐}{基史}

今年度後期の\secondGrade{}方針として以下の3点を挙げる.

\begin{itemize}
    \item 期限を意識した連絡を徹底する
    \item 確実な引き継ぎを行う
    \item 下回生の模範となる活動を行う
    \item 進捗を生む活動
\end{itemize}

\subsubsection*{連絡について}
全体行事や提出物などの期限を意識した連絡を徹底する.
また,適宜Slackなどを用いて再告知を行い,提出漏れがないように徹底する.

\subsubsection*{引き継ぎについて}
今年度後期より各局に配属される\firstGrade{}への引継ぎを確実に行う.コロナによる活動制限が解除された際に,速やかに前年度までの活動を再開できるように意識する.
文面上の情報だけでなく経験やノウハウを含めた実践的な情報を伝え,可能な限り実務経験を積む機会を\firstGrade{}にも与えるようにする.
また,各行事に関しては詳細な情報を記載した資料を作成するものとする.

\subsubsection*{模範として}
今年度前期の活動では,ライトニングトークの遅延など,\firstGrade{}の模範としてふさわしくない行動をしない様にする.や学園祭の展示企画,Advent Calendarの締め切りなどを守るように徹底しつつ,各人の活動を行っていくものとする.

\subsubsection*{進捗を生む活動}
何かしらの形で示せる成果を秋学期末までに制作する.通年ので行われたプロジェクト活動の目標を意識的に達成するよう心掛ける.
COVID-19の感染対策を十分に意識したうえで外部の学習機会を積極的に持つように行動する.獲得した知識は定例会議内のLT等で積極的に発信するようにする.

